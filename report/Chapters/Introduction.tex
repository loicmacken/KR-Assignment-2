\section{Introduction}

Cardiovascular disorders continue to be a major cause of death in the Western world. Along with all the studied physical risk factors of Coronary Heart Disease, researchers have also examined the role of psychological factors, such as personality type and psychological stress. Job strain, definable as work environment-related stress, is one of the most widely studied aspects of psycho-social stress and its importance as a risk factor for coronary heart disease remains controversial. According the World Health Organization, occupational strain is a response employees develop when presented with work demands and pressures that they are not able to cope with.  
With the aim of investigating the correlation between occupational strain and the incidence of coronary heart disease, a Bayesian Network (BN) was used to establish a framework of fourteen probabilistic variables with interactions between them. %wanna write the hypothesis
A BN is a form of statistical modeling which allows us to obtain a graphical network describing the dependencies and conditional probabilities of given propositional variables from empirical data. They have proven to be an effective tool for capturing the relationships between the factors belonging to a defined domain. 
In the upcoming sections we will describe the implementation of the BN Reasoner and the evaluation of its performance, a detailed explanation of our network, the hypothesis tested and a final discussion. 
